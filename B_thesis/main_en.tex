
%卒業論文用雛形
%\documentclass[a4j,12pt,oneside,openany]{jsbook}
% 英語なら以下を使う.
\documentclass[a4j,12pt,oneside,openany,english]{jsbook}

\usepackage{graphicx}
\usepackage{amssymb}
\usepackage{amsmath}
\usepackage{latexsym}

%jsbook を report っぽくするスタイルファイル
\usepackage{book2report}
%定理,補題,系,例題,証明などや英語用の定義がされています.
%自分なりにいじってください.
\usepackage{thesis}
% 具体的には以下のように定義されています.
% 英語の定理環境
%  \newtheorem{theorem}{Theorem}[chapter]
%  \newtheorem{lemma}{Lemma}[chapter]
%  \newtheorem{proposition}{Proposition}[chapter]
%  \newtheorem{corollary}{Corollary}[chapter]
%  \newtheorem{definition}{Definition}[chapter]
%  \newtheorem{example}{Example}[chapter]
%  \newtheorem{proof}{Proof}
% 日本語の定理環境
%  \newtheorem{theorem}{定理}[chapter]
%  \newtheorem{lemma}{補題}[chapter]
%  \newtheorem{proposition}{命題}[chapter]
%  \newtheorem{corollary}{系}[chapter]
%  \newtheorem{definition}{定義}[chapter]
%  \newtheorem{example}{例}[chapter]
%  \newtheorem{proof}{証明}
% 証明には番号をつけず,最後は Box で終わります.

% 英語で,見出しのフォントが気に入らなかったら
\renewcommand{\headfont}{\bfseries}

%ページ数が少ないときはここを大きくしてごまかそう!!効果絶大!!
\renewcommand{\baselinestretch}{1.1}

\begin{document}
%%%%%%%%%%%% 題目 %%%%%%%%%%%%%%%%%%%%%%%%%%%%%%%%%%%%%%%%%%%%%%%%%%%%%%%
%%%%%%%%%%%% ここも適当に変えてもいいと思う %%%%%%%%%%%%%%%%%%%%%%%%%%%%%%%%%
\thispagestyle{empty}
\begin{center}
\vspace*{5mm}
{\Huge {\bf 特 \hspace{12pt} 別 \hspace{12pt} 研 \hspace{12pt} 究 \hspace{12pt} 報 \hspace{12pt} 告}}\\
\vspace{2cm}
{\Large 題\hspace{8mm}目}\\
\vspace{1cm}
\underline{\LARGE{This Is}} \\
\vspace{0.5cm}
\underline{\LARGE{My Bachelor Thesis}} \\
\vspace{12mm}
{\large 指 導 教 員}\\
\vspace{6mm}
\underline{\Large Professor Jane Doe}\\
 \\
\underline{\Large Associate Professor Joe Smith}\\
\vspace{8mm}
{\large 報 告 者}\\
\vspace{6mm}
\underline{\Large John Doe}\\
\vspace{10mm}
{\Large 平成28年2月吉日}\\
\vspace{14mm}
{\Large 大阪大学基礎工学部システム科学科\\知能システム学コース}\\
\end{center}
\clearpage
\setcounter{page}{0}
\pagenumbering{roman}

%%%%%%%%%%%% 概要 %%%%%%%%%%%%%%%%%%%%%%%%%%%%%%%%%%%%%%%%%%%%%%%%%%%%
\begin{abstract}
  This is abstruct.
\end{abstract}


%%%%%%%%%%%% 目次 %%%%%%%%%%%%%%%%%%%%%%%%%%%%%%%%%%%%%%%%%%%%%%%%%%%%
\clearpage
\tableofcontents
\clearpage
\setcounter{page}{0}
\pagenumbering{arabic}

%%%%%%%%%%%% 1章 %%%%%%%%%%%%%%%%%%%%%%%%%%%%%%%%%%%%%%%%%%%%%%%%%%%
\chapter{Introduction}
Test of Theorem.

\section{First of All}

\begin{theorem}[First Theorem]
This is Theorem.
\end{theorem}

\begin{lemma}[First Lemma]
This is Lemma.
\end{lemma}

\begin{proposition}[First Proposition]
This is Proposition.
\end{proposition}

\begin{corollary}[First Corollary]
This is Corollary.
\end{corollary}

\begin{proof}
This is true, obviously.
\end{proof}

\begin{definition}[First Definition]
We call this ``That''.
\end{definition}

\begin{example}[First Example]
This is an example.
\end{example}

\section{Next of First of All}
\subsection{First Subsection}
\subsubsection{First Subsubsection}

Test of Equation.
\begin{equation}
	a = b + c
\end{equation}

\begin{table}
	\caption{Test of Table Caption}
\end{table}

\begin{figure}
	\caption{Test of Figure Caption}
\end{figure}


%%%%%%%%%%%% 2nd Chapter %%%%%%%%%%%%%%%%%%%%%%%%%%%%%%%%%%%%%%%%%%%%%%%%%%%
\chapter{2nd Chapter}

%%%%%%%%%%%% 3rd Chapter %%%%%%%%%%%%%%%%%%%%%%%%%%%%%%%%%%%%%%%%%%%%%%%%%%%
\chapter{3rd Chapter}

%%%%%%%%%%%% 4th Chapter %%%%%%%%%%%%%%%%%%%%%%%%%%%%%%%%%%%%%%%%%%%%%%%%%%%
\chapter{4th Chapter}

%%%%%%%%%%%% 5th Chapter %%%%%%%%%%%%%%%%%%%%%%%%%%%%%%%%%%%%%%%%%%%%%%%%%%%
\chapter{Conclusions}

%%%%%%%%%%%% Appendix %%%%%%%%%%%%%%%%%%%%%%%%%%%%%%%%%%%%%%%%%%%%%%%%%%%
\appendix
%%%%%%%%%%%% Appendix A %%%%%%%%%%%%%%%%%%%%%%%%%%%%%%%%%%%%%%%%%%%%%%%%%%%
\chapter{Proofs}

%%%%%%%%%%%% 謝辞 %%%%%%%%%%%%%%%%%%%%%%%%%%%%%%%%%%%%%%%%%%%
\begin{acknowledgement}
	I would like to express my deep sense of gratitude to my adviser Professor
	Toshimitsu Ushio, Graduate School of Engineering Science, Osaka University,
	for his invaluable, constructive advice and constant encouragement during this work.
	Professor Ushio's deep knowledge and his eye for detail have inspired me much.
\end{acknowledgement}

%%%%%%%%%%%% References %%%%%%%%%%%%%%%%%%%%%%%%%%%%%%%%%%%%%%%%%%%%%%%%%%
%適当に変えてねー.
\begin{thebibliography}{99}
\bibitem{a} Reference 1
\bibitem{b} Reference 2
\end{thebibliography}
%\bibliographystyle{myjunsrt}
%\bibliography{refs}

\end{document}
